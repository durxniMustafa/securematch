\section{Gesture Troubleshooting
Guide}\label{gesture-troubleshooting-guide}

This guide explains why common gesture-recognition issues occur and how
to resolve them when using \texttt{gestureClassifier.js}.

\begin{longtable}[]{@{}
  >{\raggedright\arraybackslash}p{(\linewidth - 4\tabcolsep) * \real{0.2368}}
  >{\raggedright\arraybackslash}p{(\linewidth - 4\tabcolsep) * \real{0.4211}}
  >{\raggedright\arraybackslash}p{(\linewidth - 4\tabcolsep) * \real{0.3421}}@{}}
\toprule\noalign{}
\begin{minipage}[b]{\linewidth}\raggedright
Symptom
\end{minipage} & \begin{minipage}[b]{\linewidth}\raggedright
Why it happens
\end{minipage} & \begin{minipage}[b]{\linewidth}\raggedright
What to try
\end{minipage} \\
\midrule\noalign{}
\endhead
\bottomrule\noalign{}
\endlastfoot
\textbf{No gesture fires}\texttt{deepDebug} shows
\texttt{dy}/\texttt{dp} always under ±0.08 & The default
\texttt{yawThresh}/\texttt{pitchThresh} (0.10) are higher than your
measured deltas. This happens when faces are small in the frame or on
wide‑angle cameras. & Lower the thresholds to around \textbf{0.04} or
enlarge the face area. Example:
\texttt{createClassifierMap(\{\ yawThresh:\ 0.05,\ pitchThresh:\ 0.05\ \})} \\
\textbf{Values cross the threshold but nothing emits} & Stage~2 requires
the swing to last \textbf{≥ \texttt{swingMinMs} (180~ms)} and pass
through the baseline in the opposite direction. Low‑FPS video (≤~15~FPS)
can miss that window. & Reduce \texttt{swingMinMs} (e.g.~100~ms)
\textbf{or} add a \texttt{swingMinFrames} check and allow either
condition to trigger the gesture. \\
\textbf{Opposite swing too small} & The head returns only partway to
neutral so the second crossing never reaches the threshold. & Lower
\texttt{pitchThresh}, shorten \texttt{swingMinMs}, or set
\texttt{oppSwingRatio} (e.g.~\texttt{0.4}) so a partial return swing
counts. \\
\textbf{First nod after load is ignored} & The baseline uses the pose on
the very first frame, so any initial tilt reduces the apparent movement
of the first gesture. & Hold still for a moment -- the kiosk now
auto‑calibrates after one second of steadiness -- or click
\texttt{Calibrate} manually. \\
\textbf{A few nods work, then nothing} & The adaptive baseline treats
slow motion as ``still'' and gradually drifts toward the new pose.
Future gestures then fall below the thresholds. & Increase
\texttt{baselineTol} (e.g.~0.025) to ignore casual motion or shorten
\texttt{baselineWindowMs} so baseline drift only occurs when truly
still. \\
\textbf{Shakes recognised but nods aren't (or vice versa)} & The
\textbf{axis veto} (\texttt{axisVetoFactor\ =\ 0.8}) rejects nods if
lateral motion is 80~\% of its threshold. Camera tilt mixes yaw into
pitch, causing the veto. & Lower \texttt{axisVetoFactor} (e.g.~0.4) or
level the camera. Using actual yaw/pitch angles instead of pixel deltas
avoids this coupling. \\
\end{longtable}

\textbf{Landmark deltas read as zero} \textbar{} Landmarks
234/454/10/152 have \texttt{visibility\ \textless{}\ minVis} (0.5) or
sometimes return \texttt{undefined}. \textbar{} Lower \texttt{minVis} to
0.3 and guard against \texttt{undefined}:
\texttt{if\ (!lm{[}234{]}\ \textbar{}\textbar{}\ !lm{[}454{]})\ return;}
\textbar{}
